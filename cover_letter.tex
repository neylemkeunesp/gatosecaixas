\documentclass[11pt]{letter}

\usepackage[margin=2.5cm]{geometry}
\usepackage{hyperref}

% Sender address
\address{
[Your Full Name]\\
[Your Title/Position]\\
[Your Institution]\\
[Department]\\
[Street Address]\\
[City, State/Province, ZIP/Postal Code]\\
[Country]\\
Email: [your.email@institution.edu]\\
ORCID: [0000-0000-0000-0000]
}

\signature{[Your Full Name]\\[Your Title]}

\begin{document}

\begin{letter}{
Dr. Peter Pongrácz\\
Associate Editor - Non-Farm Animals\\
Applied Animal Behaviour Science\\
Department of Ethology\\
Eötvös Loránd University\\
Budapest, Hungary\\
Email: peter.pongracz@ttk.elte.hu
}

\opening{Dear Dr. Pongrácz,}

I am pleased to submit our manuscript entitled \textbf{"Box-Seeking Behaviour in Felidae: An Integrative Ethological Review of Proximate Mechanisms, Ontogeny, Adaptive Function, and Phylogenetic Conservation"} for consideration as a Review Article in \textit{Applied Animal Behaviour Science}.

\textbf{Rationale and Significance}

This comprehensive review addresses a fundamental yet understudied aspect of felid ethology: the attraction to confined spaces, particularly cardboard boxes. While this behaviour is widely observed across domestic cats and captive felids, its mechanistic basis and adaptive significance have received limited systematic analysis. Our manuscript fills this gap by employing Tinbergen's four-question framework to integrate evidence from stress physiology, evolutionary ecology, thermoregulation, and veterinary behavioural medicine.

The manuscript is particularly suitable for \textit{AABS} for three reasons:

\textbf{1. Direct relevance to AABS scope:} Our review synthesizes empirical evidence from controlled studies published in \textit{AABS}, most notably the seminal Vinke et al. (2014) randomized trial demonstrating stress reduction in shelter cats provided with hiding boxes. We extend this work by contextualizing it within broader ethological theory and comparative biology.

\textbf{2. Applied ethology focus:} Consistent with the journal's mission, we emphasize practical applications for animal welfare. We detail evidence-based protocols for veterinary clinics, animal shelters, and multi-cat households, demonstrating how fundamental ethological insights translate directly into welfare improvements.

\textbf{3. Methodological rigor and theoretical integration:} We critically evaluate existing evidence, identify methodological limitations, and propose testable hypotheses for future research. The Tinbergen framework provides a structured approach rarely applied to companion animal behaviour, offering a model for future reviews.

\textbf{Key Contributions}

Our manuscript makes several novel contributions:

\begin{itemize}
    \item \textbf{Integrative framework:} First comprehensive application of Tinbergen's four questions to box-seeking behaviour, synthesizing proximate mechanisms (stress reduction, thermoregulation) with ultimate causation (predation strategies, anti-predator defences).
    
    \item \textbf{Phylogenetic perspective:} We argue that box-seeking represents a conserved felid trait maintained by strong selective pressures, not a domestication artifact. This reframes the behaviour as a fundamental ethological need.
    
    \item \textbf{Clinical translation:} We provide evidence-based recommendations for hiding box implementation in veterinary and shelter settings, supported by randomized controlled trials and physiological data.
    
    \item \textbf{Research agenda:} We identify six critical knowledge gaps requiring investigation, including phylogenetic comparative studies, neurobiological substrates, and optimal environmental parameters.
\end{itemize}

\textbf{Originality and Ethical Compliance}

This manuscript represents original work that has not been published previously and is not under consideration elsewhere. As a literature review, no animals were used in the preparation of this manuscript. All cited studies involving animals adhered to ethical guidelines of their respective institutions and the International Society for Applied Ethology.

We confirm that all authors have approved the manuscript for submission and agree to its publication in \textit{Applied Animal Behaviour Science}. There are no conflicts of interest to declare.

\textbf{Suggested Reviewers}

We respectfully suggest the following reviewers with expertise in feline behaviour, applied ethology, and animal welfare:

\textbf{1. Dr. Dennis C. Turner}  
University of Zurich, Switzerland  
Email: turnerd@access.uzh.ch  
\textit{Expertise: Feline social behaviour, human-cat relationships, domestication}

\textbf{2. Dr. Sarah Ellis}  
International Cat Care, UK  
Email: sarah.ellis@icatcare.org  
\textit{Expertise: Feline welfare, environmental enrichment, shelter cat behaviour}

\textbf{3. Dr. Lauren Finka}  
Nottingham Trent University, UK  
Email: lauren.finka@ntu.ac.uk  
\textit{Expertise: Feline personality assessment, welfare science, shelter environments}

\textbf{4. Dr. Kristyn Vitale}  
Oregon State University, USA  
Email: kristyn.vitale@oregonstate.edu  
\textit{Expertise: Cat cognition, social behaviour, human-cat attachment}

\textbf{5. Dr. Judi Stella}  
The Ohio State University, USA  
Email: stella.15@osu.edu  
\textit{Expertise: Feline stress physiology, environmental enrichment, welfare}

None of these individuals have conflicts of interest with the authors or this manuscript.

\textbf{Compliance with Journal Guidelines}

Our manuscript adheres to all \textit{AABS} author guidelines:

\begin{itemize}
    \item Abstract: 398 words (limit: 400)
    \item Introduction: approximately 750 words (limit: 750)
    \item Highlights: 5 bullet points (85 characters maximum each)
    \item References: Author-year format using natbib
    \item Formatting: 12pt Times font, double-spaced, line-numbered
    \item Ethics statement, funding declaration, and conflict of interest statement included
\end{itemize}

\textbf{Conclusion}

We believe this manuscript will be of substantial interest to the \textit{AABS} readership, including applied ethologists, veterinary behaviourists, animal welfare scientists, and shelter professionals. By integrating mechanistic and evolutionary perspectives with clinical applications, we provide a comprehensive framework for understanding box-seeking behaviour and its implications for feline welfare.

We look forward to your editorial decision and welcome reviewer feedback to strengthen the manuscript.

Thank you for considering our submission.

\closing{Sincerely,}

\end{letter}

\end{document}
