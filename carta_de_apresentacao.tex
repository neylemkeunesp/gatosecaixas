\documentclass[11pt]{letter}

\usepackage[margin=2.5cm]{geometry}
\usepackage{hyperref}

% Endereço do remetente
\address{
Prof. Dr. Ney Lemke\\
Coordenador de Tecnologia da Informação; Professor do Instituto de Biociências\\
Universidade Estadual Paulista “Júlio de Mesquita Filho” (UNESP)\\
Departamento de Física e Biofísica – Instituto de Biociências (IBB/UNESP)\\
Rua Prof. Dr. Antônio Celso Wagner Zanin, 250\\
Botucatu, SP, 18618-689\\
Brasil\\
Email: ney.lemke@unesp.br\\
ORCID: 0000-0002-2797-941X
}

\signature{Prof. Dr. Ney Lemke\\[Coordenador de Tecnologia da Informação; Professor do Instituto de Biociências]}

\begin{document}

\begin{letter}{
Dr. Peter Pongrácz\\
Editor Associado - Animais Não-Fazenda\\
Applied Animal Behaviour Science\\
Departamento de Etologia\\
Universidade Eötvös Loránd\\
Budapeste, Hungria\\
Email: peter.pongracz@ttk.elte.hu
}

\opening{Prezado Dr. Pongrácz,}

Tenho o prazer de submeter nosso manuscrito intitulado \textbf{"Comportamento de Busca por Caixas em Felidae: Uma Revisão Etológica Integrativa dos Mecanismos Próximos, Ontogenia, Função Adaptativa e Conservação Filogenética"} para consideração como Artigo de Revisão na \textit{Applied Animal Behaviour Science}.

\textbf{Justificativa e Significância}

Esta revisão abrangente aborda um aspecto fundamental, porém pouco estudado, da etologia felina: a atração por espaços confinados, especialmente caixas de papelão. Embora este comportamento seja amplamente observado em gatos domésticos e felinos em cativeiro, sua base mecanicista e significância adaptativa receberam análise sistemática limitada. Nosso manuscrito preenche essa lacuna empregando o quadro das quatro questões de Tinbergen para integrar evidências da fisiologia do estresse, ecologia evolutiva, termorregulação e medicina comportamental veterinária.

O manuscrito é particularmente adequado para a \textit{AABS} por três razões:

\textbf{1. Relevância direta para o escopo da AABS:} Nossa revisão sintetiza evidências empíricas de estudos controlados publicados na \textit{AABS}, mais notavelmente o ensaio randomizado seminal de Vinke et al. (2014) que demonstrou a redução do estresse em gatos de abrigos com o fornecimento de caixas para se esconderem. Estendemos este trabalho contextualizando-o dentro da teoria etológica mais ampla e da biologia comparativa.

\textbf{2. Foco em etologia aplicada:} Consistente com a missão da revista, enfatizamos as aplicações práticas para o bem-estar animal. Detalhamos protocolos baseados em evidências para clínicas veterinárias, abrigos de animais e lares com múltiplos gatos, demonstrando como insights etológicos fundamentais se traduzem diretamente em melhorias no bem-estar.

\textbf{3. Rigor metodológico e integração teórica:} Avaliamos criticamente as evidências existentes, identificamos limitações metodológicas e propomos hipóteses testáveis para futuras pesquisas. O quadro de Tinbergen fornece uma abordagem estruturada raramente aplicada ao comportamento de animais de companhia, oferecendo um modelo para futuras revisões.

\textbf{Principais Contribuições}

Nosso manuscrito faz várias contribuições inéditas:

\begin{itemize}
    \item \textbf{Quadro integrativo:} Primeira aplicação abrangente das quatro questões de Tinbergen ao comportamento de busca por caixas, sintetizando mecanismos próximos (redução de estresse, termorregulação) com causas últimas (estratégias de predação, defesas anti-predador).
    
    \item \textbf{Perspectiva filogenética:} Argumentamos que a busca por caixas representa um traço felino conservado, mantido por fortes pressões seletivas, e não um artefato da domesticação. Isso reenquadra o comportamento como uma necessidade etológica fundamental.
    
    \item \textbf{Tradução clínica:} Fornecemos recomendações baseadas em evidências para a implementação de caixas de esconderijo em ambientes veterinários e de abrigos, apoiadas por ensaios clínicos randomizados e dados fisiológicos.
    
    \item \textbf{Agenda de pesquisa:} Identificamos seis lacunas críticas de conhecimento que requerem investigação, incluindo estudos comparativos filogenéticos, substratos neurobiológicos e parâmetros ambientais ótimos.
\end{itemize}

\textbf{Originalidade e Conformidade Ética}

Este manuscrito representa um trabalho original que não foi publicado anteriormente e não está sob consideração em outro lugar. Como uma revisão da literatura, nenhum animal foi usado na preparação deste manuscrito. Todos os estudos citados envolvendo animais aderiram às diretrizes éticas de suas respectivas instituições e da Sociedade Internacional de Etologia Aplicada.

Confirmamos que todos os autores aprovaram o manuscrito para submissão e concordam com sua publicação na \textit{Applied Animal Behaviour Science}. Não há conflitos de interesse a declarar.

\textbf{Revisores Sugeridos}

Respeitosamente, sugerimos os seguintes revisores com expertise em comportamento felino, etologia aplicada e bem-estar animal:

\textbf{1. Dr. Dennis C. Turner}  
Universidade de Zurique, Suíça  
Email: turnerd@access.uzh.ch  
\textit{Expertise: Comportamento social felino, relações humano-gato, domesticação}

\textbf{2. Dr. Sarah Ellis}  
International Cat Care, Reino Unido  
Email: sarah.ellis@icatcare.org  
\textit{Expertise: Bem-estar felino, enriquecimento ambiental, comportamento de gatos em abrigos}

\textbf{3. Dr. Lauren Finka}  
Nottingham Trent University, Reino Unido  
Email: lauren.finka@ntu.ac.uk  
\textit{Expertise: Avaliação da personalidade felina, ciência do bem-estar, ambientes de abrigo}

\textbf{4. Dr. Kristyn Vitale}  
Oregon State University, EUA  
Email: kristyn.vitale@oregonstate.edu  
\textit{Expertise: Cognição felina, comportamento social, apego humano-gato}

\textbf{5. Dr. Judi Stella}  
The Ohio State University, EUA  
Email: stella.15@osu.edu  
\textit{Expertise: Fisiologia do estresse felino, enriquecimento ambiental, bem-estar}

Nenhum desses indivíduos tem conflitos de interesse com os autores ou este manuscrito.

\textbf{Conformidade com as Diretrizes da Revista}

Nosso manuscrito adere a todas as diretrizes para autores da \textit{AABS}:

\begin{itemize}
    \item Resumo: 398 palavras (limite: 400)
    \item Introdução: aproximadamente 750 palavras (limite: 750)
    \item Destaques: 5 pontos (máximo de 85 caracteres cada)
    \item Referências: formato Autor-ano usando natbib
    \item Formatação: fonte Times 12pt, espaçamento duplo, linhas numeradas
    \item Declaração de ética, declaração de financiamento e declaração de conflito de interesses incluídas
\end{itemize}

\textbf{Conclusão}

Acreditamos que este manuscrito será de grande interesse para os leitores da \textit{AABS}, incluindo etologistas aplicados, veterinários comportamentalistas, cientistas de bem-estar animal e profissionais de abrigos. Ao integrar perspectivas mecanicistas e evolutivas com aplicações clínicas, fornecemos um quadro abrangente para a compreensão do comportamento de busca por caixas e suas implicações para o bem-estar felino.

Aguardamos sua decisão editorial e agradecemos o feedback dos revisores para fortalecer o manuscrito.

Obrigado por considerar nossa submissão.

\closing{Atenciosamente,}

\end{letter}

\end{document}
